\documentclass[UTF8,10pt,a4paper,twoside,fleqn,openany]{ctexbook}
\title{基于流形学习子空间的人脸识别算法}
\author{姜希成}
\date{\today}

\usepackage{amsmath}
\usepackage{ctex}

\begin{document}
\maketitle
\newpage

\renewcommand\thesection{\Roman{section}} 
\renewcommand\thesubsection{\roman{subsection}}

\paragraph{【摘要】}
{局部线性嵌入(LLE)是一种众所周知的流形学习方法,正交邻域保持投影(ONPP)是LLE的线性扩展,
稀疏线性嵌入(SLE)进一步扩展了基于LLE的稀疏方法,稀疏内核嵌入(SKE)的概念与SLE类似,
只是操作在内核空间上。本文实验证明稀疏学习框架具有更好的学习能力,特别是在样本量小的情况下。}
\paragraph{【关键字】}
{降维,弹性网,图像识别,流形学习,稀疏投影。}

\section{介绍}

简单介绍:

LLE

ONPP

SLE

SKE

本文其余部分安排如下,在第二部分中将详细介绍LLE,ONPP,
第三部分详细介绍SLE,第四部分中简单介绍SKE,
第五部分中对SLE和SKE算法进行评估,第六部分给出结论。


\section{准备工作}
在本节中,将详细介绍LLE,ONPP算法。

矩阵$ X=[x_{1},x_{2},...,x_{N}] $是数据矩阵,包括所有的训练样本$ {\{x_{i}\}}_{i=1}^{N} \in R^{m}$。维度m通常是很大的,(稀疏)线性降维的目标是变换数据从原来的高维空间到低维空间。
$$ y=A^{T}x \in R^{d}  \eqno(1) $$ 

	\subsection{局部线性嵌入(LLE)}
	
	\subsection{正交领域保持投影(ONPP)}

\section{稀疏线性嵌入(SLE)}

\section{稀疏内核嵌入(SKE)}

\section{实验}

\section{结论}


\begin{thebibliography}{99}
	\bibitem{1}
	
\end{thebibliography}


\end{document}